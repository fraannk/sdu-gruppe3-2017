\chapter*{Forord}\label{chap:forord}
\addcontentsline{toc}{chapter}{Forord}

Dette projekt er udarbejdet af studerende på 4. semester elektronik og datateknik, ved Syddansk Universitet, Teknisk Fakultet, foråret 2017. Projektet er en sammenfatning af de opnåede fagligheder fra semesterets undervisning. Afleveringen af rapporten er dertil et krav for at kunne gå til eksamen.\\

Som udgangspunkt antages det at læseren har et tilsvarende fagligt niveau som en 4. semester studerende med linjefag indenfor elektronik og datateknik.
Rapporten tager derfor ikke hensyn til at skulle forklare den grundliggende viden, en studerende med samme faglige niveau antages at have.

\subsection{Læsevejledning}
Denne rapport er opbygget således, at den læses fra start til slut. Rapportens struktur er opdelt i hovedsektioner og undersektioner. Opbygningen er taksonomisk både i kapitelstrukturer og som helhed. Der vil i starten af hvert hovedafsnit være en figur som viser hvilken del af produktet der i det respektive afsnit bliver dækket.

\subsection{Arbejdsfordeling}
Arbejdet har været opdelt groft på den kapitelinddeling som rapporten er opbygget efter. Kapitlet om "Analoge antialiasing og rekonstruktions filtre" er udarbejdet af Jonas Jensen Holmgren. Kapitlet om "Hardware implementering og operativ systemer" er ud arbejdet af Jörn Jacobi. Kapitlet om "Digitale filtre og signalbehandling" er udarbejdet af Dennis Amtoft Jensen og Kenneth Petersen. Kapitlet om "Test af equalizer" er udarbejdet af Simon Møller, Søren Frank og Dennis Amtoft Jensen. Endeligt er kapitlerne "Indledningen", "Diskussion" og "Konklusion" færdigstillet i plenum.

\subsection{Typografiske konventioner}
Her er en kort oversigt over de typografiske konventioner der anvende i denne rapport\\
\begin{tabular}{l p{0.6\linewidth}}
	\textit{Kursiv tekst}			& Angiver filnavne i den tilhørende kodebase samt fremhævelse af ord eller fagudtryk. \\
	\textbf{Fed tekst}				& Bruges til a fremhæve produkt eller system specifikke betegnelser.\\
	\texttt{Konstant brede tekst}	& Anvendes til kildekode eksempler. Ligeledes anvendes afgrænsende områder.\\
	\emph{Fremhævet tekst}		    & Bliver brugt når der gives en kort introduktion til hvert kapitel.\\
\end{tabular}

\subsection{Typografi}
Rapporten har en opbygning der gør den behagelig at læse og indeholder derfor mange tomme sider der er angivet med sidetal.
Det samlede antal sider med udgør derfor kun XXX hele sider. 
\husk{ALLE}{tælle det endelige side antal til forord.}

%\note{Hvem har skrevet (og evt. udgivet) rapporten - og hvornår?
%	Hvorfor er den blevet skrevet - hvad er dens baggrund, og hvem er dens målgruppe?
%	Hvordan er den blevet til? (Yderst kort: arbejdsmetodik, afgrænsning, evt. projektperiode)
%	Hvad er formålet med rapporten? Hvem er den skrevet til gavn for, og hvad skal de kunne bruge den til?
%	Eventuel tak til vejledere, faglige hjælpere, økonomiske bidragydere osv.}
%
%\note{Forordet skal handle om selve rapporten. Det skal give læseren et billede af, hvad det egentlig er for et værk, han eller hun sidder med:}

