\chapter*{Forord}\label{chap:forord}
\addcontentsline{toc}{chapter}{Forord}




Forordet skal handle om selve rapporten. Det skal give læseren et billede af, hvad det egentlig er for et værk, han eller hun sidder med:

\begin{itemize}
	\item Hvem har skrevet (og evt. udgivet) rapporten - og hvornår?
	\item Hvorfor er den blevet skrevet - hvad er dens baggrund, og hvem er dens målgruppe?
	\item Hvordan er den blevet til? (Yderst kort: arbejdsmetodik, afgrænsning, evt. projektperiode)
	\item Hvad er formålet med rapporten? Hvem er den skrevet til gavn for, og hvad skal de kunne bruge den til?
	\item Eventuel tak til vejledere, faglige hjælpere, økonomiske bidragydere osv.
\end{itemize}

Forordet står selvstændigt, allerførst - som regel endda før indholdsfortegnelsen. Man kan med en vis ret sige, at forordet slet ikke er en del af selve rapporten; det er en slags "etiket".
