\chapter*{Forord}\label{chap:forord}
\addcontentsline{toc}{chapter}{Forord}

Dette projekt er udarbejdet af studerende på 4. semester elektronik og datateknik, ved Syddansk Universitet, Teknisk Fakultet, foråret 2017. Projektet er en sammenfatning af de opnåede fagligheder fra semesterets undervisning. Afleveringen af rapporten er dertil et krav for at kunne gå til eksamen.

\jj{Beskrive hvem vi forventer at læseren er til}

\subsection{Læsevejledning}
Denne rapport er opbygget således, at den læses fra start til slut. Rapportens struktur er opdelt i hovedsektioner og undersektioner. Opbygningen er taksonomisk både i kapitelstrukturer og som helhed. Der vil i starten af hvert hovedafsnit være en figur som viser hvilken del af produktet der i det respektive afsnit bliver dækket.

\subsection{Typografiske konventioner}
Her er en kort oversigt over de typografiske konventioner der anvende i denne rapport
\\
\textit{Kursiv tekst} : Angiver filnavne i den tilhørende kodebase.
\\
\textbf{Fed tekst} : Bruges til a fremhæve produkt eller system specifikke betegnelser.
\\
\texttt{Konstant brede tekst} : Anvendes til kildekode eksempler.
\\
\emph{Fremhævet tekst} : Bliver brugt når der gives en kort introduktion til hvert kapitel. 
\\
\jj{Gennemgang af alle typografier og at de bliver brugt rigtigt igennem rapporten.}

Semesterprojektet er udarbejdet af studerende på 4. semester elektronik og datateknik, ved Syddansk Universitet, Teknisk Fakultet, foråret 2017. Projektet er en sammenfatning af de opnåede fagligheder fra semesterets undervisning. Afleveringen af rapporten er dertil et krav for at kunne gå til eksamen.
\\ \\
\husk{Kenneth}{Noget med arbejdsfordelingen}

\note{Hvem har skrevet (og evt. udgivet) rapporten - og hvornår?
	Hvorfor er den blevet skrevet - hvad er dens baggrund, og hvem er dens målgruppe?
	Hvordan er den blevet til? (Yderst kort: arbejdsmetodik, afgrænsning, evt. projektperiode)
	Hvad er formålet med rapporten? Hvem er den skrevet til gavn for, og hvad skal de kunne bruge den til?
	Eventuel tak til vejledere, faglige hjælpere, økonomiske bidragydere osv.}

\note{Forordet skal handle om selve rapporten. Det skal give læseren et billede af, hvad det egentlig er for et værk, han eller hun sidder med:}

