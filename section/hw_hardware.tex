\section{Hardware lag}
Konintroduktion til de kommende underafsnit.

Forskel mellem design og release opsætning af hardware.
Global.h switches

\subsection{PWM}
Gennegang af PWM på EMP.
PWM til projekt.

Noter til PWM : PWM opløsning, fase korrect signal. Se om der kan findes information mht. en PWM frekvens på ca. 10 gange signal frekvensen.


Segway til DAC.

Kom kort ind på de uhensigtmæssige signal fejl ved PWM i forhold til måling og bestemmelse af EQ's effektivitet.

\subsection{DAC}

Sikring af værdier med for høje niveauer må ikke overstige 4096.

Argument for valg af komponent, indsvingningstid, båndbrede, 12bit.

\subsection{SPI}

Overensstemmelse mellem valg komponent protokol og SPI ( her bruges Freescale ).
Beskriv 

\subsection{ADC}

Adc afsnit skal komme her lige efter SPI/DAC ( således overgang fra lyd input til lyd output)


\subsection{UART}
Meget kort beskrivelse af opsætning af uart til den ønskede baudrate.

\subsection{LCD}
LCD konfiguratoin er arvet fra EMP board, men det er i drift ikke noget krav at der bruges et LCD, således bør det kunne fjernes ved et release build og kun brugbart ved udvikling

\subsection{TIVA}
Kort beskrivelse af Tiva Board konfiguration, med henvisninger til realse note for produktet.

\subsection{FPU}
Beskrivelse af produkt specifik aktivering af FPU med henvisning til TI dokumentation. 

\subsection{Interrupt}


\subsection{sysTick og Timers}


\subsection{ADC fra Simon}
En analog til digital konverter kan anvendes til at konvertere et sammenhængende analogt spændingssignal til et diskret digitalt nummer. TMC4C123GH6PM microcontrolleren har to identiske 12 bit A/D konvertere indbygget - ADC0 og ADC1. Det konverterede signal, kan derefter behandles vha. digitale signalbehandlingsmetoder. 
Da der ønskes at opbygge en stereostyret equalizer, bruges begge A/D konvertere. Konfigurationen er derfor ens på hhv. ADC0 og ADC1. Begge A/D konvertere kører uafhængigt af hinanden. Afhængig af hvilken Sample frequencer der er valgt, gemmes
resultatet af A/D konverteringen i et FIFO register (first in - first out). Da der kun er brug for én sample ad gangen, vælges ud fra databladet at konfigurere A/D konverterne med Sample sequencer 3, og derved gemmes resultatet af konverteringen også i ADCSSFIFO3 registeret for hhv. ADC0 og ADC1.
Da det ikke er alle porte, der kan anvendes på microcontrolleren til A/D konvertering, er pin PB5 valgt til at styre venstre kanal, og pin PE4 er valgt til at styre højre kanal. Herudover kan den også køre i mono, frem for stereo - denne er konfigureret på pin PE5. Der ønskes en samplingsfrekvens på $44,1\si\kilo\hertz$ for at undgå aliasing. Denne samplingsfrekvens er bestemt ud fra Nyquist frekvensen. For at få denne samplingsfrekvens korrekt, skal microcontrollerens CPU frekvens sættes til $80\si\mega\hertz$. Der bruges et PWM signal, som er interruptstyret. Dette signal skal køre et bestemt antal cycles.

\begin {equation}
\text{Cycles} = \frac{\text{CPU frekvens}}{\text{Sample frekvens}} => \frac{80\cdot 10^6\si\hertz}{44100\si\hertz} = 1814 \text{cycles}
\end {equation}

Hver gang PWM'en har kørt i det beregnede antal cycles, bliver et interrupt initialiseret, og samplingen starter samt gemmer værdien. Ved næste interrupt, starter processen for A/D konverteringen igen.
Da A/D konverterne er opløst i 12 bit, og den maksimale spænding der bliver registreret er $3,3\si\volt$ - vil maksimalværdien for konverteringen være $4096$. 