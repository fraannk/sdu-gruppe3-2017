\section{IIR - Infinte Impulse Response}\label{sec:iir} 

Et IIR filters differensligning er givet ved ligning (\ref{eq:iir_diff_lign}).
Hvor $a_i$ og $b_i$ er koefficienterne fra det digitale filter, $n$ er index af det givne sample,
$M =N + 1$ hvor $N$ er filterets orden.

\begin{align}
y(n) = \sum\limits_{i=0}^{M} b_i x(n-i) - \sum\limits_{i=1}^{N} a_i y(n-i) 
\label{eq:iir_diff_lign}
\end{align}


    Overføringsfunktionen for et IIR-filter er på formen fra ligning (\ref{eq:dig_tf}). Filterdesignet består  
    i at finde koefficienterne for denne funktion.
    
    \begin{align}
        H(z) = \dfrac{Y(z)}{X(z)} = \dfrac{\sum\limits_{i=0}^M b_i z^{-i}}{1 + \sum\limits_{i=1}^N a_i z^{-i}}
        \label{eq:dig_tf}
    \end{align}


Som løsningsmodel med IIR-filtre anvendes den Bilinear Transformations metode og består af følgende trin:
    \begin{enumerate}
        \item Definer de digital specifikationer. Kompenser mod forwrængning til de analoge specifikationer.
        \item Udfør analog filter design.
        \item Substituer BLT for at få det digitale filters overføringsfunktion og heraf koefficienterne. 
    \end{enumerate}



    \subsection{Frekvenpreswarp}
    Da center frekvensen af filteret vil blive forskudt når der udføres en Bilinear
    transformation er det nødvendigt at prewarp'e (kompensere mod forvrængning), således at center frekvensen passer i det digitale filter.\\
    $\Omega_0$ er den analoge centerfrekvens, $\omega_0$ er den ønskede centerfrekvens, $f_s$ er samplingsfrekvensen. 
    Forholdet mellem $\Omega_0$ og $\omega_0$:
  
    \begin{align}
    \omega_0 &= \dfrac{2 \pi f_0}{f_s} \label{eq:freq_warp_w0}\\
    \Delta \omega &= \dfrac{\omega_0}{Q} = \omega_2 - \omega_1 \label{eq:freq_warp_BW} \\ 
    \omega_0^2 &= \omega_1 \cdot \omega_2 \label{eq:freq_warp_w1_w2} \\
    \Omega &= \tan\left(  \dfrac{\omega}{2} \right) \label{eq:freq_warp}
    \end{align}
        
    Vha. ligning (\ref{eq:freq_warp_w0}), (\ref{eq:freq_warp_BW}),(\ref{eq:freq_warp_w1_w2}),(\ref{eq:freq_warp}) 
    kommer et udtryk for den prewarp'et båndbredde $\Delta \Omega$ i 
    forhold til den ønskede båndredde $\Delta \omega$ og den analog center frekvens $\Omega_0^2$.
\begin{align}
    \tan \left( \dfrac{\Delta \omega}{2} \right) = \tan \left( \dfrac{\omega_2 - \omega_1}{2} \right) &= \dfrac{\tan(\omega_2/2) - tan(\omega_1/2)}{1 + tan(\omega_1/2) \cdot tan(\omega_2/2)} = \dfrac{\Omega_2 - \Omega_1}{1 + \Omega_1 \Omega_2} = \dfrac{\Delta \Omega}{1 + \Omega_0^2}\\
    \iff \Delta \Omega &= (1 + \Omega_0^2) \tan \left( \dfrac{\Delta \omega}{2} \right)
\end{align}

    \subsection{Anlog filter }

    Et analogt filter $H_a(s)$ er givet ved ligning (\ref{eq:analog_tf}). Ved at finde amplituden $|H_a (j\omega)|$ og sætte den 
    lig med den ønskede forstærkning $G_C$ ved knækfrekvensen $f_C$ kan der findes et matematisk udtryk der beskriver de ønskede parametre.
    
    Forstærkningen $G_C$ defineres til at være den goemetriske middelværdi mellem den ønskede forstærkning $G$ og reference forstærkningen $G_0$.
\husk{Dennis}{skal omformuleres}  

    \begin{align}
        H_a(s) &= \dfrac{b_i s^i + b_{i-1} s^{i-1}+ \dots + b_0 }{a_i s^i + a_{i-1} s^{i-1} +  \dots +a_0 }
        \label{eq:analog_tf} \\
        G_C &= \dfrac{G_0^2 + G^2}{2} \label{eq:analog_gc_def} \\
        |H(\Omega_C)|^2 &= G_C^2 \label{eq:analog_amp}
    \end{align}
    
    \subsection{Bilinear Transformation}
    Når det analog filter skal transformeres fra s-plan til z-plan bruges ligning (\ref{eq:s_to_z}). Hvilket giver et 
    filter i z-planet der har samme orden som det analog filter. Derudover bliver faseresponset af det digitale filter mangen til 
    det analog filter hvilket medfører at der bliver ulineært fasedrejning på filteret. 
    Heraf kommer koefficienterne $a_1, \dots, a_N, b_0,\dots, b_N $. 
  
    \begin{align}
    s = \dfrac{z^{-1} - 1}{z^{-1} + 1}  
    \label{eq:s_to_z}
    \end{align}
    \begin{align}
    H(z) = H(s)\big|_{s = \frac{z^{-1} - 1}{z^{-1}+ 1 }}
    \end{align}

    
   \subsection{Equalizer}

   En equalizers kan opdeles i en række bånd hvori hvert bånd repræsenterer et 
   2. ordens filter (biquad). Ved at sætte $G_0 = 1$ ($0 dB$) kan filtrene kaskadekobles hvilket giver 
   et resulterende filter $H_{res}$ som er produktet af de individuelle filtre $H_i$. \\
   Dette kan ses i ligning (\ref{eq:kaskade_iir}) hvor $K$ er antal bånd af equalizeren.  

   \begin{align}
    H_{res}(z) = \prod_{i=1}^{K} H_i(z)
    \label{eq:kaskade_iir}
   \end{align} 




 






