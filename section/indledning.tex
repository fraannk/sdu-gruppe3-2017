\chapter{Indledning}
\note{
Musik er en naturlig del af hverdagen. Om det er baggrundsmusikken nede i Brugsen, til Grøn koncert eller i høretelefonerne i bussen, så er den altid til stede. At musikken er en så væsentlig størrelse giver anledning til at undersøge hvilke optimeringsmåder man kan foretage sig, for at få den bedst mulige oplevelse ud af det - hver især. En equalizers egenskab er at udforme en musikafspillers lydsignal, så musikken lyder som brugeren finder bedst.
I denne rapport vil læseren blive guidet gennem processen bag ved udvikling af en digital equalizer - fra musikafspillerens udgangssignal til det ønskede lydsignal.
}
\husk{Kenneth}{Kan være denne rubrik bare skal stå som et chapter med dertilhørende underemner}
\section{Formål}
\note{Indledningen skal ikke handle om rapporten, men om rapportens emne. Man kan sige, at den kridter banen op ved at præsentere emneområdet.
	\\
	* Hvilket overordnet emne tager rapporten udgangspunkt i? \\
	* Hvilket specifikt problem vil man derfor tage fat i og kigge nærmere på? }
%
%Formålet med projektet er, at forene de opnåede fagligheder fra undervisningen, samt at træne den studerende i, at kunne forstå og forklare afgrænsede systemer med funktionalitet der relaterer til faglighederne i Indlejrede Systemer og Signalbehandling, på 4. semester Elektronik og Datateknik. Fagene fra semesteret der indgår i projektet er Embedded programmering, Analoge filtre og signaler og Operativsystemer. Ledningsteori som er et mindre fag under Analoge filtre, er ikke med i rapporten, da undervisningen ligger så sent i semesteret, at der ikke er tid til at se på implementation af faget i rapporten.
%
%Semesterets fagbeskrivelse vil blive brugt som tjekliste, for at komme igennem så meget af stoffet som muligt.
Formålet med dette projekt, er at formidle de opnåede fagligheder fra 4. semesters undervisning på uddannelsen diplomingeniør i elektronik og datateknik. Semesteret hovedemne omhandler Indlejrede systemer og signalbehandling, hvor fagene Analoge filter og signaler, Digital signalbehandling, Embedded programmering og Operativsystemer inkludere. Der skal på et grundlæggende niveau, kunne relateres til faglighederne i ét samlet produkt ud fra disse. For at kunne håndtere dette, vil semesterets fagbeskrivelse blive anvendt til at opnå det ønskede formål.
\husk{Simon}{jeg har udkommenteret tidligere skrevet formål, og skrevet noget nyt.}


\section{Problemformulering}
%\note{Generel fortælling om hvordan man kan finde frem til en projekt der kan blive dækket af de fagligheder de skal omhandle}
%
%\note{Specefike problemer der kan være ved at lave en equalzier som et embedded system}
%
%\note{Hvilke spørgsmål kan der stilles om man ønsker at få besvaret ?}

%Ud fra formålet med projektet, er der blevet valgt, at udvikle en digital equalizer, da den dækker centrale punkter af fagbeskrivelsen. De væsentligste punkter som projektet dækker er analog og digital filterteori, signalbehandling og strukturelle opbygning af et operativ system.

%\note{Problemformulering: Hvilket spørgsmål vil denne rapport helt præcist stille skarpt på, og hvilke delspørgsmål vil det måske være nødvendigt at besvare for at kunne besvare hovedspørgsmålet?}
%
%Hovedspørgsmålet som der vil blive fokuseret på er:
%\begin{enumerate}
%	\item Hvilken digital signalbehandling kan give den ønskede signal filtrering?
%\end{enumerate}
%Herunder ses underspørgsmålene som vil blive brugt til at besvare hovedspørgsmålet:
%\begin{enumerate}
%	\item Kan en digital equalizer realiseres ved hjælp af en Tiva™ TM4C123GH6PM Microcontroller?
%	\item Kan EMP boarded fra Embedded Programmering anvendes i projektet?
%	\item 
%\end{enumerate}
%\note{
%Projektet går ud på at designe en parametrisk equalizer, der kan efterbehandle et lydsignal. Den skal realiseres ved hjælp af en Tiva™ TM4C123GH6PM Microcontroller som er platformen i faget Embedded Programmering. Equalizeren skal have nogle forudbestemte bånd, og samtidig have et bånd som brugeren frit kan modulere.
%}
Ud fra formålet med projektet, er der blevet valgt at fremstille et produkt, hvoraf alle fagligheder kan anvendes. Produktet der ønskes realiseret, er at konstruere en digital equalizer. Denne digitale equalizer, skal realiseres vha. en Tiva™ TM4C123GH6PM Microcontroller, som er platformen i faget Embedded Programmering. Equalizeren skal have nogle forudbestemte bånd, og samtidig have et bånd som brugeren frit kan modulere. Ud fra fagbeskrivelsen, kan dette relateres i et så bredt omfang, at formålet opfyldes. Problemformuleringen for fremstilling af den digitale equalizer lyder: 
\begin{itemize}
	\item Hvordan realiseres en digital equalizer vha. en Tiva™ TM4C123GH6PM microcontroller?
	\item Hvordan anvendes digital signalbehandling, til at give de ønskede båndspecifikationer?
	\item Hvordan anvendes EMP boarded fra Embedded programmering til en digital equalizer?
\end{itemize}

\husk{Simon}{Rettet problemformulering til - kunne godt bruge andres øjne på. Jeg har udkommenteret tidligere.}

\section{Proces- og arbejdsmetode}
\note{Metodebeskrivelse: Hvordan vil det blive gjort? (Opgaven bygger på den-og-den litteratur samt interviews med dem og dem. Først undersøger vi dét, og så undersøger vi dét. Derefter …)}

Gennemarbejdningen med denne rapport vil tage udgangspunkt i undervisningen med dertilhørende litteratur. Ekspertviden vil blive hentet fra underviserne, med henblik på bedre forståelse af stoffet.

Rapporten er bygget op omkring signalvejen igennem equalizeren. Det kan derved lette forståelsen for, hvordan signalet ændres løbende igennem systemet.
\husk{Kenneth}{Tage udgangspunkt i signalvejen - men bedre ord for det}


\section{Projektafgrænsning}
\note{Hvilke emner / delemner kommer rapporten ikke ind på? Hvorfor?\\
	Det er meget almindeligt, at problemformulering og metoderedegørelse simpelthen er en del af indledningen, men så bør de normalt være markeret med underrubrikker (under-overskrifter) for overskuelighedens skyld. Hvis du er uddannelsessøgende, så tjek, om skolen eller institutionen (eller din vejleder) har særlige krav på det punkt.}

En equalizer kan realiseres på analogt og digitalt. I dette projektet vil den blive realiseret digitalt, da Embedded Programmering skal indgå. Der træffes endvidere valget om at gøre den parametrisk, for bedre fleksibilitet. 



\section{Kravspecifikation} \label{afs:kravspecifikation}
I dette afsnit fremstår de opsatte krav for projektet, der er fastsat dels ud fra kravet til selve projektbeskrivelsen, men også ud fra de studerende selv med henblik på at efterligne den virkelige ingeniørverden.

\begin{itemize}
	\item Prototypen udvikles på Tiva™ C Series TM4C123G LaunchPad Evaluation Kit.
	\item Tiva™ TM4C123G serie Microcontroller anvendes.
	\item Kildekoden er en del af produktet og skal overholde EMP-C standarden.
	\item Equalizeren skal virke som parametrisk equalizer.
	\item Det skal være muligt at ændre frekvensprofiler på equalizeren.
	\item Indgangs- og udgangssignaler skal overholde line-level standard på, nominal $-10 dBV$, $0,894 V_{pp}$ (Forbruger elektronik line-signal).
	\item Tilgængelige filtertyper i frekvensprofiler - Højpass, Lavpass, Båndpass og Båndstop.
	\item Equalizerparametre skal omfatte - Frekvens, båndbredde og gain.
	\item Equalizerens funktionalitet og effektivitet efterprøves med et velbeskrevet test-signal.
\end{itemize}

\section{Løsningsmodel}
Løsningsmodellen som ses på figur \ref{fig:losningsmodel}, er den overordnede arbejdsproces faget digital signalbehandling, hvoraf meget af stoffet bliver hentet fra. Modellen gør det nemmere at dele arbejdet op, da det er et blokdiagram.

\begin{figure}[h]
	\centering
	\includegraphics[width=15cm]{billeder/flow_losn}
	\caption{Blokdiagram over signalvejen}
	\label{fig:losningsmodel}
\end{figure}

\section{Læsevejledning}
Denne rapport er opbygget...
\husk{Kenneth}{Noget med signalvejen og opbygningen af rapporten}