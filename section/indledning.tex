\chapter{Indledning}

\section{Formål}
\note{Indledningen skal ikke handle om rapporten, men om rapportens emne. Man kan sige, at den kridter banen op ved at præsentere emneområdet.
	\\
	* Hvilket overordnet emne tager rapporten udgangspunkt i? \\
	* Hvilket specifikt problem vil man derfor tage fat i og kigge nærmere på? }

Formålet med projektet er at fremstille en parametrisk equalizer, der kan efterbehandle et lydsignal.
Projektet indeholder elementer af fagligheder fra 4. semester for Indlejrede systemer og signalbehandling på Elektronik og Datateknik med elementer af Digital signalbehandling, Embedded programmering, Analoge filtre og signaler og Operativsystemer.




\section{Problemformulering}
\note{Problemformulering: Hvilket spørgsmål vil denne rapport helt præcist stille skarpt på, og hvilke delspørgsmål vil det måske være nødvendigt at besvare for at kunne besvare hovedspørgsmålet?}



\section{Proces- og arbejdsmetode}
\note{Metodebeskrivelse: Hvordan vil det blive gjort? (Opgaven bygger på den-og-den litteratur samt interviews med dem og dem. Først undersøger vi dét, og så undersøger vi dét. Derefter …)}

\section{Projektafgrænsning}
\note{Hvilke emner / delemner kommer rapporten ikke ind på? Hvorfor?\\
	Det er meget almindeligt, at problemformulering og metoderedegørelse simpelthen er en del af indledningen, men så bør de normalt være markeret med underrubrikker (under-overskrifter) for overskuelighedens skyld. Hvis du er uddannelsessøgende, så tjek, om skolen eller institutionen (eller din vejleder) har særlige krav på det punkt.}



\section{Kravspecifikation} \label{afs:kravspecifikation}
\begin{itemize}
	\item Prototypen udvikles på Tiva™ C Series TM4C123G LaunchPad Evaluation Kit.
	\item Tiva™ TM4C123GH6PM Microcontroller anvendes.
	\item Kildekoden er en del af produktet og skal overholder EMP-C standarden.
	\item Equalizeren skal være en parametrisk equalizer.
	\item Det skal være muligt at uploade frekvensprofiler til equalizeren.
	\item Indgangs- og udgangssignaler skal overholde line-level standard på, nominal $-10 dBV$, $0,894 V_{pp}$ (Forbruger elektronik line-signal).
	\item Tilgængelige filtertyper i frekvensprofiler - Højpass, Lavpass, Båndpass og Båndstop.
	\item Equalizerparametre skal omfatte - Frekvens, båndbredde og gain.
	\item Der kan anvendes maksimalt 6 bånd i frekvensprofilerne (mono) og 3 bånd (stereo)\footnote{Angivet krav er et groft estimat}.   
	\item Equalizerens funktionalitet og effektivitet efterprøves med et velbeskrevet test-signal.
\end{itemize}

\section{Løsningsmodel}

\section{Læsevejledning}