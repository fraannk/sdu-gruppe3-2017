\chapter{Konklusion} \label{kap:konklusion}
Det konkluderes at:
\begin{itemize}
	\item En digital equalizer kan realiseres både ved hjælp af FIR og IIR-designmetoder. I dette projekt blev IIR valgt som løsningsmodel.
	\item De ønskede båndspecifikationer fås ved bilineær transformation.
	\item Analoge filtre implementeres som anti-aliasingsfilter på indgangen og rekonstruktionsfilter på udgangen. Butterworth-typologien blev valgt som kompromis mellem kompleksitet og krav til fasedrejning.
	\item Kravene til de analoge filtre er: 
	\begin{itemize}
		\item Der skal være tilstrækkelig dæmpning til, at aliasing ikke forekommer. 
		\item Ændringen af gruppeløbetiden må ikke overstige $2\si{\milli\second}$.
	\end{itemize}
	\item Det implementerede realtidsoperativsystem med en lagdelt/monolitisk arkitektur, giver den nødvendige funktionalitet og fleksibilitet der muliggør processering af lydsignalet inden for de stillede krav. 
\end{itemize}
