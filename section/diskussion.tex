\chapter{Diskussion og vurdering}\label{kap:diskussion}


* Forskel i målt og teoretisk fasedrejning i det samlede analoge filtersystem
* Årsag til valg af butterworth frem for bessel. 
* Linear fasedrejning i forhold til minimum fasedrejning.

* Fejl ved båndbredde grundet bilinear transformation 
* Beregningstyngden af de to filtertyper i forhold til hardwarefunktionalitet og cpu-load. 

* Muligheden for fejl ved missede systicks i RTCS
* Forskel mellem PWM og DAC generede lyd-signaler, og hvordan de analoge filtre bliver påvirket af PWM signalet.
* Fordele ved at anvende hurtigere og større mikrocontrollere

* Indgangs- og udgangsimpedans, linjestandarder, større afvigelse inden for audio-verdenen, mange subjektive holdninger
* 


\husk{Søren}{diskussion ang. faktorer og trial-and-error}





Før equalizeren blev testet, blev der diskuteret hvilke målinger, der var nødvendige at foretage, for at få de mest beskrivende resultater af hvordan alle funktionaliterne i det samlede system virkede.
Det ønskedes at finde ud af, om koden til DSP modulet stemte overens med de teoretisk beregnede filtre i Matlab. 
Heraf blev der dannet grundlag for, hvilke målinger der ønskedes målt.
De første målinger stemte ikke overens med de teoretiske, hvorfor metoden "trial-and-error" blev anvendt. 
Heraf blev offset fejlen manuelt korrigeret.
Dette beviste dog ikke, om koden i DSP modulet var korrekt ift. teorien.
Der blev da diskuteret om koefficienterene i DSP modulet var korrekte, hvilket førte til sammenligning med de teoretiske.
Det viste sig, at koden var korrekt og stemte overens med teorien - den endelige fejl blev da yderligere diskuteret.
Hardwaren blev da sat i fokus, men der blev aldrig vurderet om dette var den endelig fejl.





 


\begin{itemize}
\item Fixed point beregning.
\item Assembler optimerede funktioner.
\end{itemize}