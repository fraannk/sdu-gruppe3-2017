\section{Opsummering}
Der er mange aspekter af det design der ligger til grund for den realiserede equalizer der kunne forbedres.
Arbejdet og udviklings processen har dog bidraget med en større indsigt i både hvad der har har været hensigtsmæssigt at gøre, og hvor problemer er opstået.
Har har der helt klart været tale om løbende forbedringer og implementering som den viden der skulle til at løse forskellige problemstillinger blev lært eller stillet tilrådighed.\\

Hvis projektet, som produkt skulle videreudvikles, ville de næste trin af udviklingen ligge i retning af
\begin{itemize}
	\item At udskrifte operativ systemet fra det nuværende til et der robust system som fx freeRTOS med en preemtive arkitektur.
	\item Udbygge fleksibiliteten i shell'en således at det kunne være muligt, løbende at konfigurer og definere equalizer profiler
	\item Indbygge fx I2C kommunikation, således at systemet kan bruges som modul i andre og større lydsystemer.
\end{itemize} 

andre punkter
\begin{itemize}
	\item Latency ?
	\item Nøjagtighed
	\item 
\end{itemize}

\subsection{Delkonklusion}
Som helhed blev den ønskede implementering opnået.
Systemet er funktionelt og kan inden for en rimelig fleksibilitet håndtere den korrekte signalbehandling.

