\section{FIR - Finite Impulse Response}\label{sec:fir}
Et FIR filter består ligesom det tidligere beskrevet IIR filter af 

Input og output signalet er givet ved foldning og er opskrevet i ligning \ref{fir_ligning},
\begin {equation} 
\sum\limits_{i=0}^{N-1} h(k)x(n-k) \label{fir_ligning}
\end {equation}
hvor $h(k)$ er impulsresponset medsat IIR filterne har en endelig længde, $N$, antal værdier. FIR filterets impulsrespons er et set af filter koeffecienter. Sendes en impuls ind i filteret bestående af et $"1"$ efterfulgt af mange $"0"er$, bliver filterets output sættet af koeffecienter $"1"$-samplen kører igennem. 


\begin{itemize}
	\item Frekvenssampling beskrevet i dybden
	\item Windowsmetoden
	\item Equiriple
\end{itemize}


